\section{Methoden und Durchführung}
\subsection{Ausbreitungsgeschwindigkeit und relative Permittivität}
Zuletzt sollen in zwei verschiedenen Koaxialkabeln stehende Wellen erzeugt und die Resonanzfrequenzen, bei denen die stehenden Wellen beobachtet werden, gemessen werden. Aus den Abständen zwischen den Resonanzen soll anschließend mit \cref{eq:c} die Ausbreitungsgeschwindigkeit der Welle im Kabel berechnet werden.
\begin{equation}
	c = 2l(f_{n+1} - f_n)
	\label{eq:c}
\end{equation}
Zuletzt soll dann mit \cref{eq:er} die relative Permittivität bestimmt werden.
\begin{equation}
	\epsilon_r = \left( \frac{c_0}{c}\right) ^2
	\label{eq:er}
\end{equation}

\section{Auswertung}
\subsection{Ausbreitungsgeschwindigkeit und relative Permittivität}
Im letzten Abschnitt werden Lichtgeschwindigkeit und Permittivität zwei verschiedener Koaxialkabel bestimmt. In \cref{gleich} sind die Messergebnisse für die Messung mit Kabel 1 sowohl für offenes Kabelende als auch für kurzgeschlossenes Kabelende dargestellt. Dabei ist die Leistung in dBm gegen die Frequenz in GHz aufgetragen.