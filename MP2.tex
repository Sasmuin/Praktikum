% Autor: Simon May
% Datum: 2017-10-05
% Diese Datei bietet ein minimalistisches Grundgerüst für ein LaTeX-Dokument,
% z.B. für die Bearbeitung der Aufgaben.
\documentclass[
	% Papierformat
	a4paper,
	% Schriftgröße (beliebige Größen mit „fontsize=Xpt“)
	12pt,
	% Schreibt die Papiergröße korrekt ins Ausgabedokument
	pagesize,
	% Sprache für z.B. Babel
	ngerman
]{scrartcl}

% Achtung: Die Reihenfolge der Pakete kann (leider) wichtig sein!
% Insbesondere sollten (so wie hier) babel, fontenc und inputenc (in dieser
% Reihenfolge) als Erstes und hyperref und cleveref (Reihenfolge auch hier
% beachten) als Letztes geladen werden!

% Silbentrennung etc.; Sprache wird durch Option bei \documentclass festgelegt
\usepackage{babel}
% Verwendung der Zeichentabelle T1 (Sonderzeichen etc.)
\usepackage[T1]{fontenc}
% Legt die Zeichenkodierung der Eingabedatei fest, z.B. UTF-8
\usepackage[utf8]{inputenc}
% Schriftart
\usepackage{lmodern}
% Zusätzliche Sonderzeichen
\usepackage{textcomp}

% Mathepaket (intlimits: Grenzen über/unter Integralzeichen)
\usepackage[intlimits]{amsmath}
% Ermöglicht die Nutzung von \SI{Zahl}{Einheit} u.a.
\usepackage{siunitx}
% Zum flexiblen Einbinden von Grafiken (\includegraphics)
\usepackage{graphicx}
% Abbildungen im Fließtext
\usepackage{wrapfig}
% Abbildungen nebeneinander (subfigure, subtable)
\usepackage{subcaption}
% Funktionen für Anführungszeichen
\usepackage{csquotes}
% Zitieren, Bibliographie
\usepackage{biblatex}

% Verlinkt Textstellen im PDF-Dokument
\usepackage[unicode]{hyperref}
% "Schlaue" Referenzen (nach hyperref laden!)
\usepackage{cleveref}

% siunitx: Deutsche Ausgabe, Messfehler getrennt mit ± ausgeben
\sisetup{
	locale=DE,
	separate-uncertainty
}

\begin{document}
\begin{titlepage}
	\centering
	{\scshape\LARGE Versuchsbericht zu \par}
	\vspace{1cm}
	{\scshape\huge E4\par}
	\vspace{2.5cm}
	{\LARGE Gruppe 2 Mo\par}
	\vspace{0.5cm}
	{\large Nils Kulawiak (E-Mail: n\_kula01@wwu.de) \par}
	{\large Oliver Brune (E-Mail: o\_brun02@wwu.de) \par}
	\vfill
	durchgeführt am 15.1.2017\par
	
	\vfill

	{\large \today\par}
\end{titlepage}


\tableofcontents
	
	
\newpage
\section{Einleitung}
Ziel dieses Versuches ist es die Phasenumwandlung einer Kupfer-Gold Mischung für verschiedene Temperaturen zu untersuchen, wobei bekannt ist, dass es sich um ein fcc-Gitter handelt. Im ersten Teil des Versuchs wird das erreicht, indem diese Mischung unter sich verändernden Winkel mit Röntgenstrahlung bestrahlt wird und die Intensität dazu gemessen wird. Wenn es zu einer konstruktiven Interferenz kommt, kann mithilfe der Bragg-Gleichung auf den Abstand der Ebenen voneinander geschlossen werden:
\begin{equation}
2d sin(\theta) = n \lambda
\label{bragg}
\end{equation}

Wenn bekannt ist um was für eine Ebene es sich handelt, kann daraus wiederum der Gitterabstand eines kubischen Gitters berechnet werden.

\begin{equation}
d = \frac{a}{\sqrt{h^{2}+k^{2}+l^{2}}}
\end{equation}

Damit lässt sich daraufhin zeigen, wie sich der Gitterabstand mit der Temperatur verändert, wobei ein linearer Anstieg erwartet wird.

Außerdem soll versucht werden die sogenannte Fernordnung zu bestimmen, welche aussagt wie geordnet ein Kristallgitter ist. Ein fcc-Gitter lässt sich in vier verschiedene sc-Gitter aufteilen. Bei einer vollständigen Ordnung befinden sich alle Goldatome auf dem sc-Gitter mit der Basis (0,0,0), während sich die Kupferatome auf den anderen drei sc-Gitter befinden.
Bei einer vollständigen Unordnung ist es rein zufällig, wie die Gold und Kupferatome verteilt sind.

Um die Fernordnung zu bestimmen, muss zuerst die Intensität theoretisch bestimmt werden. 
Dies passiert bis auf eine Konstante genau mit
\begin{equation}
I_{hkl} \propto |F_{hkl}|^{2} \cdot p \cdot L_{p} \cdot D_{T}.
\end{equation}

\begin{itemize}


\item Der Flächenhäufigkeitsfaktor p beschreibt dabei wie häufig eine Ebene in der Gitterstruktur vorkommt.

\item Der Lorentz-Polarisationsfaktor $L_{p}$ korrigiert verschiedene Effekte der Röntgenbeugung und beträgt $L_{p} =\frac{1 + cos^{2}(2 \theta)}{sin^{2}(\theta) cos(\theta)}$. 

\item Der Debye-Waller-Faktor beschreibt den Einfluss der Wärmebewegung auf die Intensität und beträgt $D_{T} = \text{exp}(-2B \cdot \frac{sin^{2}(\theta)}{\lambda^{2}})$, wobei B ein experimentell bestimmter Wert ist.

\item Der Strukturfaktor $F_{hkl}$ beschreibt das Streuvermögen einer Elementarzelle.
\end{itemize}

Um den Streufaktor zu berechnen wird über alle Basisatome summiert.
\begin{equation}
F_{hkl} \propto \sum_{i=1}^{n} f_{i} e^{2 \pi i(hx_{i} + ky_{i} + lz_{i})}
\end{equation}

Da sich die Basisatome bei $(0,0,0) ; (\frac{a}{2},\frac{a}{2},0) ; (\frac{a}{2},0,\frac{a}{2})$ und $(0, \frac{a}{2}, \frac{a}{2})$ befinden, ergibt sich der Streufaktor bei vollständiger Ordnung zu 
\begin{equation}
F_{hkl} = f_{Au} + f_{Cu} (e^{2 \pi i(h+l)} + e^{2 \pi i(k+l)} + e^{2 \pi i(h+k)}).
\end{equation}

Somit ergibt sich für jede Ebene ein Peak, allerdings sind die Peaks mit nur geraden oder ungeraden h,k und l deutlich größer, da der Strukturfaktor dort $f_{Au} + 3f_{Cu}$ ergibt und sonst $f_{Au} - f_{Cu}$.

Anders verhält es sich bei der vollständigen Unordnung. Dort ergibt sich der Strukturfaktor zu 
\begin{equation}
F_{hkl} = (\frac{1}{4} f_{Au} + \frac{3}{4}f_{Cu}) (1 + e^{2 \pi i(h+l)} + e^{2 \pi i(k+l)} + e^{2 \pi i(h+k)}).
\end{equation}

Somit bleibt der Strukturfaktor gleich für alle h,k und l gerade oder ungerade, allerdings ergibt sich der Strukturfaktor für alle anderen Ebenen zu 0. 
Die Röntgenreflexe, die immer unabhängig von der Fernordnung erscheinen, nennt man Fundamentalreflex, während die sich ändernden Überstrukturreflex genannt werden.

Unter der Annahme, dass der Strukturfaktor linear mit der Fernordnung $\mu$ skaliert ergibt er sich zu
\begin{equation}
F_{hkl} = \left \{ \begin{array}{ll}
f_{Au} + 3f_{Cu} \text{  für nur gerade oder ungerade h,k,l} \\
\mu (f_{Au} - f_{Cu}) \text{  sonst} \\
\end{array} \right .
\end{equation}

Daraus lässt sich dann die Fernordnung berechnen  mit 
\begin{equation}
\mu^{2} = \frac{I^{Ü}(\mu)}{I^{Ü}(1)} = \frac{I^{Ü}_{exp}(\mu)\cdot I^{F}_{theo}}{I^{F}_{exp} \cdot I^{Ü}_{theo}(1)}.
\end{equation}  

Außerdem sollen noch die Kernaussagen von Debye überprüft werden. Diese heißen:
1. Die Schärfe der Interferenzmaxima wird durch die Wärmebewegung nicht beeinflusst.

2. Durch die Wärmebewegung ändert sich die räumliche Intensitätsverteilung.

3. Aufgrund der Wärmebewegung nimmt die Interferenzintensität
\begin{itemize}
\item mit zunehmender Temperatur 
\item mit zunehmendem Winkelabstand zwischen Einfalls- und Beobachtungsrichtung 
\item mit abnehmender Wellenlänge
\end{itemize}
exponentiell ab.
\section{Erhöhung der Temperatur}
\subsection{Ergebnis}
Um herauszuf

\begin{figure}[h]
	\centering
	\includegraphics[scale=0.35]{25.PNG}
	\caption{Intensitätsverteilung bei Raumtemperatur}
	\label{ski}
\end{figure}

\end{document}
