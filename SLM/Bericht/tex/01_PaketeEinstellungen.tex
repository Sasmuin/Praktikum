% Autor: Simon May
% Datum: 2017-10-04

% --- Pakete einbinden
% --- Pakete erweitern LaTeX um zusätzliche Funktionen.
%     Dies ist ein Satz nützlicher Pakete.

% Silbentrennung etc.; Sprache wird durch Option bei \documentclass festgelegt
\usepackage{babel}
\usepackage{ifthen}
\if LuaTeX
	% Schriftart (Latin Modern)
	\usepackage{fontspec}
	\fontspec{Latin Modern Roman}
\else
	% Verwendung der Zeichentabelle T1 (für Sonderzeichen etc.)
	\usepackage[T1]{fontenc}
	% Legt die Eingabe-Zeichenkodierung fest, z.B. UTF-8
	\usepackage[utf8]{inputenc}
	% Schriftart (Latin Modern)
	\usepackage{lmodern}
	% Zusätzliche Sonderzeichen
	\usepackage{textcomp}
\fi

% Nutzen von +, -, *, / in \setlength u.ä. (z.B. \setlength{\a + 3cm})
\usepackage{calc}
% Wird benötigt, um \ifthenelse zu benutzen
\usepackage{xifthen}
% Optionen für eigene definierte Befehle
\usepackage{xparse}

% Verbessertes Aussehen des Schriftbilds durch kleine Anpassungen
\usepackage{microtype}
% Automatische Formatierung von Daten
\usepackage[useregional]{datetime2}
% Wird für Kopf- und Fußzeile benötigt
\usepackage{scrlayer-scrpage}
% Einfaches Wechseln zwischen unterschiedlichen Zeilenabständen
\usepackage{setspace}
% Optionen für Listen (enumerate, itemize, …)
\usepackage{enumitem}
% Automatische Anführungszeichen
\usepackage{csquotes}
% Zusätzliche Optionen für Tabellen (tabular)
\usepackage{array}

% Mathepaket (intlimits: Grenzen über/unter Integralzeichen)
\usepackage[intlimits]{amsmath}
% Mathe-Symbole, \mathbb etc.
\usepackage{amssymb}
% Weitere Mathebefehle
\usepackage{mathtools}
% „Schöne“ Brüche im Fließtext
\usepackage{xfrac}
% Ermöglicht die Nutzung von \SI{Zahl}{Einheit} u.a.
\usepackage{siunitx}
% Definition von Unicode-Symbolen; Nach [utf8]inputenc laden!
\usepackage{newunicodechar}
% Unicode-Formeln mit pdfLaTeX
% Autor: Simon May
% Datum: 2015-03-04

% Diese Datei ermöglicht es, Mathe-Symbole (z.B. \gamma) direkt als
% Sonderzeichen (d.h. γ) einzugeben

% silence unterdrückt Warnungen; vor hyperref laden
\usepackage{silence}
\WarningFilter[pdflatex-unicode-math]{newunicodechar}{Redefining Unicode character}
\ActivateWarningFilters[pdflatex-unicode-math]

\newunicodechar{†}{\dag}
\newunicodechar{‡}{\ddag}
\newunicodechar{…}{\ldots}
\newunicodechar{⋯}{\cdots}
\newunicodechar{⋮}{\vdots}
\newunicodechar{⋱}{\ddots}
\newunicodechar{⋰}{\iddots}
\newunicodechar{α}{\alpha}
\newunicodechar{β}{\beta}
\newunicodechar{γ}{\gamma}
\newunicodechar{δ}{\delta}
\newunicodechar{ε}{\varepsilon}
\newunicodechar{ϵ}{\epsilon}
\newunicodechar{ζ}{\zeta}
\newunicodechar{η}{\eta}
\newunicodechar{θ}{\theta}
\newunicodechar{ϑ}{\vartheta}
\newunicodechar{ι}{\iota}
\newunicodechar{κ}{\kappa}
\newunicodechar{ϰ}{\varkappa}
\newunicodechar{λ}{\lambda}
\newunicodechar{μ}{\mu}
\newunicodechar{ν}{\nu}
\newunicodechar{ξ}{\xi}
\newunicodechar{ο}{o}
\newunicodechar{π}{\pi}
\newunicodechar{ρ}{\rho}
\newunicodechar{ϱ}{\varrho}
\newunicodechar{σ}{\sigma}
\newunicodechar{τ}{\tau}
\newunicodechar{υ}{\upsilon}
\newunicodechar{φ}{\varphi}
\newunicodechar{ϕ}{\phi}
\newunicodechar{χ}{\chi}
\newunicodechar{ψ}{\psi}
\newunicodechar{ω}{\omega}
\newunicodechar{Α}{\mathrm{A}}
\newunicodechar{Β}{\mathrm{B}}
\newunicodechar{Γ}{\Gamma}
\newunicodechar{Δ}{\Delta}
\newunicodechar{Ε}{\mathrm{E}}
\newunicodechar{Ζ}{\mathrm{Z}}
\newunicodechar{Η}{\mathrm{H}}
\newunicodechar{Θ}{\Theta}
\newunicodechar{Ι}{\mathrm{I}}
\newunicodechar{Κ}{\mathrm{K}}
\newunicodechar{Λ}{\Lambda}
\newunicodechar{Μ}{\mathrm{M}}
\newunicodechar{Ν}{\mathrm{N}}
\newunicodechar{Ξ}{\Xi}
\newunicodechar{Ο}{\mathrm{O}}
\newunicodechar{Π}{\Pi}
\newunicodechar{Ρ}{\mathrm{P}}
\newunicodechar{Σ}{\Sigma}
\newunicodechar{Τ}{\mathrm{T}}
\newunicodechar{Υ}{\Upsilon}
\newunicodechar{Φ}{\Phi}
\newunicodechar{Χ}{\Chi}
\newunicodechar{Ψ}{\Psi}
\newunicodechar{Ω}{\Omega}
\newunicodechar{∑}{\sum}
\newunicodechar{∫}{\int}
\newunicodechar{∬}{\iint}
\newunicodechar{∭}{\iiint}
\newunicodechar{⨌}{\iiiint}
\newunicodechar{∮}{\oint}
\newunicodechar{∯}{\oiint}
\newunicodechar{∰}{\oiiint}
\newunicodechar{∇}{\nabla}
\newunicodechar{∂}{\partial}
\newunicodechar{√}{\sqrt}
\newunicodechar{∈}{\in}
\newunicodechar{∋}{\ni}
\newunicodechar{∉}{\notin}
\newunicodechar{∀}{\forall}
\newunicodechar{∃}{\exists}
\newunicodechar{∄}{\nexists}
\newunicodechar{∴}{\therefore}
\newunicodechar{∵}{\because}
\newunicodechar{〈}{\langle}
\newunicodechar{〉}{\rangle}
\newunicodechar{⌊}{\lfloor}
\newunicodechar{⌋}{\rfloor}
\newunicodechar{⌈}{\lceil}
\newunicodechar{⌉}{\rceil}
\newunicodechar{∼}{\sim}
\newunicodechar{∝}{\propto}
\newunicodechar{∞}{\infty}
\newunicodechar{ℵ}{\aleph}
\newunicodechar{ℏ}{\hbar}
\newunicodechar{℘}{\wp}
\newunicodechar{ℓ}{\ell}
\newunicodechar{∅}{\emptyset}
\newunicodechar{×}{\times}
\newunicodechar{⋅}{\cdot}
\newunicodechar{÷}{\div}
\newunicodechar{⋆}{\star}
\newunicodechar{∘}{\circ}
\newunicodechar{⋄}{\diamond}
\newunicodechar{⊕}{\oplus}
\newunicodechar{⊖}{\ominus}
\newunicodechar{⊗}{\otimes}
\newunicodechar{⊘}{\oslash}
\newunicodechar{⊙}{\odot}
\newunicodechar{±}{\pm}
\newunicodechar{∓}{\mp}
\newunicodechar{≈}{\approx}
\newunicodechar{≡}{\equiv}
\newunicodechar{≠}{\ne}
\newunicodechar{≥}{\ge}
\newunicodechar{≤}{\le}
\newunicodechar{≫}{\gg}
\newunicodechar{≪}{\ll}
\newunicodechar{⊂}{\subset}
\newunicodechar{⊃}{\supset}
\newunicodechar{⊆}{\subseteq}
\newunicodechar{⊇}{\supseteq}
\newunicodechar{⊈}{\nsubseteq}
\newunicodechar{⊉}{\nsupseteq}
\newunicodechar{≔}{\coloneqq}
\newunicodechar{≕}{\eqqcolon}
\newunicodechar{¬}{\neg}
\newunicodechar{∨}{\vee}
\newunicodechar{∧}{\wedge}
\newunicodechar{∪}{\cup}
\newunicodechar{∩}{\cap}
\newunicodechar{⋁}{\bigvee}
\newunicodechar{⋀}{\bigwedge}
\newunicodechar{⋃}{\bigcup}
\newunicodechar{⋂}{\bigcap}
\newunicodechar{⟂}{\perp}
\newunicodechar{∥}{\parallel}
\newunicodechar{∦}{\nparallel}
\newunicodechar{𝚤}{\imath}
\newunicodechar{𝚥}{\jmath}
\newunicodechar{⇔}{\Leftrightarrow}
\newunicodechar{⇕}{\Updownarrow}
\newunicodechar{⇐}{\Leftarrow}
\newunicodechar{⇒}{\Rightarrow}
\newunicodechar{⇑}{\Uparrow}
\newunicodechar{⇓}{\Downarrow}
\newunicodechar{↔}{\leftrightarrow}
\newunicodechar{↕}{\updownarrow}
\newunicodechar{←}{\leftarrow}
\newunicodechar{→}{\rightarrow}
\newunicodechar{↑}{\uparrow}
\newunicodechar{↓}{\downarrow}
\newunicodechar{⟷}{\longleftrightarrow}
\newunicodechar{⟵}{\longleftarrow}
\newunicodechar{⟶}{\longrightarrow}
\newunicodechar{⇇}{\leftleftarrows}
\newunicodechar{⇉}{\rightrightarrows}
\newunicodechar{⇈}{\upuparrows}
\newunicodechar{⇊}{\downdownarrows}
\newunicodechar{⟺}{\Longleftrightarrow}
\newunicodechar{⟸}{\Longleftarrow}
\newunicodechar{⟹}{\Longrightarrow}
\newunicodechar{↦}{\mapsto}
\newunicodechar{↤}{\mapsfrom}
\newunicodechar{⟼}{\longmapsto}
\newunicodechar{⟻}{\longmapsfrom}
\newunicodechar{⟾}{\Longmapsto}
\newunicodechar{⟽}{\Longmapsfrom}
\newunicodechar{↗}{\nearrow}
\newunicodechar{↖}{\nwarrow}
\newunicodechar{↘}{\searrow}
\newunicodechar{↙}{\swarrow}
\newunicodechar{↩}{\hookleftarrow}
\newunicodechar{↪}{\hookrightarrow}
\newunicodechar{↶}{\curvearrowleft}
\newunicodechar{↷}{\curvearrowright}
\newunicodechar{↺}{\circlearrowleft}
\newunicodechar{↻}{\circlearrowright}
\newunicodechar{↫}{\looparrowleft}
\newunicodechar{↬}{\looparrowright}
\newunicodechar{⇋}{\leftrightharpoons}
\newunicodechar{⇌}{\rightleftharpoons}
\newunicodechar{↼}{\leftharpoonup}
\newunicodechar{↽}{\leftharpoondown}
\newunicodechar{⇀}{\rightharpoonup}
\newunicodechar{⇁}{\rightharpoondown}
\newunicodechar{↿}{\upharpoonleft}
\newunicodechar{↾}{\upharpoonright}
\newunicodechar{⇃}{\downharpoonleft}
\newunicodechar{⇂}{\downharpoonright}
\newunicodechar{𝔸}{\mathbb{A}}
\newunicodechar{𝔹}{\mathbb{B}}
\newunicodechar{ℂ}{\mathbb{C}}
\newunicodechar{𝔻}{\mathbb{D}}
\newunicodechar{𝔼}{\mathbb{E}}
\newunicodechar{𝔽}{\mathbb{F}}
\newunicodechar{𝔾}{\mathbb{G}}
\newunicodechar{ℍ}{\mathbb{H}}
\newunicodechar{𝕀}{\mathbb{I}}
\newunicodechar{𝕁}{\mathbb{J}}
\newunicodechar{𝕂}{\mathbb{K}}
\newunicodechar{𝕃}{\mathbb{L}}
\newunicodechar{𝕄}{\mathbb{M}}
\newunicodechar{ℕ}{\mathbb{N}}
\newunicodechar{𝕆}{\mathbb{O}}
\newunicodechar{ℙ}{\mathbb{P}}
\newunicodechar{ℚ}{\mathbb{Q}}
\newunicodechar{ℝ}{\mathbb{R}}
\newunicodechar{𝕊}{\mathbb{S}}
\newunicodechar{𝕋}{\mathbb{T}}
\newunicodechar{𝕌}{\mathbb{U}}
\newunicodechar{𝕍}{\mathbb{V}}
\newunicodechar{𝕎}{\mathbb{W}}
\newunicodechar{𝕏}{\mathbb{X}}
\newunicodechar{𝕐}{\mathbb{Y}}
\newunicodechar{ℤ}{\mathbb{Z}}
\newunicodechar{𝒜}{\mathcal{A}}
\newunicodechar{ℬ}{\mathcal{B}}
\newunicodechar{𝒞}{\mathcal{C}}
\newunicodechar{𝒟}{\mathcal{D}}
\newunicodechar{ℰ}{\mathcal{E}}
\newunicodechar{ℱ}{\mathcal{F}}
\newunicodechar{𝒢}{\mathcal{G}}
\newunicodechar{ℋ}{\mathcal{H}}
\newunicodechar{ℐ}{\mathcal{I}}
\newunicodechar{𝒥}{\mathcal{J}}
\newunicodechar{𝒦}{\mathcal{K}}
\newunicodechar{ℒ}{\mathcal{L}}
\newunicodechar{ℳ}{\mathcal{M}}
\newunicodechar{𝒩}{\mathcal{N}}
\newunicodechar{𝒪}{\mathcal{O}}
\newunicodechar{𝒫}{\mathcal{P}}
\newunicodechar{𝒬}{\mathcal{Q}}
\newunicodechar{ℛ}{\mathcal{R}}
\newunicodechar{𝒮}{\mathcal{S}}
\newunicodechar{𝒯}{\mathcal{T}}
\newunicodechar{𝒰}{\mathcal{U}}
\newunicodechar{𝒱}{\mathcal{V}}
\newunicodechar{𝒲}{\mathcal{W}}
\newunicodechar{𝒳}{\mathcal{X}}
\newunicodechar{𝒴}{\mathcal{Y}}
\newunicodechar{𝒵}{\mathcal{Z}}
\newunicodechar{𝕬}{\mathfrak{A}}
\newunicodechar{𝕭}{\mathfrak{B}}
\newunicodechar{𝕮}{\mathfrak{C}}
\newunicodechar{𝕯}{\mathfrak{D}}
\newunicodechar{𝕰}{\mathfrak{E}}
\newunicodechar{𝕱}{\mathfrak{F}}
\newunicodechar{𝕲}{\mathfrak{G}}
\newunicodechar{𝕳}{\mathfrak{H}}
\newunicodechar{𝕴}{\mathfrak{I}}
\newunicodechar{𝕵}{\mathfrak{J}}
\newunicodechar{𝕶}{\mathfrak{K}}
\newunicodechar{𝕷}{\mathfrak{L}}
\newunicodechar{𝕸}{\mathfrak{M}}
\newunicodechar{𝕹}{\mathfrak{N}}
\newunicodechar{𝕺}{\mathfrak{O}}
\newunicodechar{𝕻}{\mathfrak{P}}
\newunicodechar{𝕼}{\mathfrak{Q}}
\newunicodechar{𝕽}{\mathfrak{R}}
\newunicodechar{𝕾}{\mathfrak{S}}
\newunicodechar{𝕿}{\mathfrak{T}}
\newunicodechar{𝖀}{\mathfrak{U}}
\newunicodechar{𝖁}{\mathfrak{V}}
\newunicodechar{𝖂}{\mathfrak{W}}
\newunicodechar{𝖃}{\mathfrak{X}}
\newunicodechar{𝖄}{\mathfrak{Y}}
\newunicodechar{𝖅}{\mathfrak{Z}}

\DeactivateWarningFilters[pdflatex-unicode-math]


% Farben
\usepackage{xcolor}
% Einbinden von Grafiken (\includegraphics)
\usepackage{graphicx}
% .tex-Dateien mit \includegraphics einbinden
\usepackage{gincltex}
% Größere Freiheiten bei Dateinamen mit \includegraphics
\usepackage{grffile}
% Abbildungen im Fließtext
\usepackage{wrapfig}
% Zitieren, Bibliographie (Biber als Bibliographie-Programm verwenden!)
\usepackage[style=verbose, backend=biber]{biblatex}
% Abbildungen nebeneinander (subfigure, subtable)
\usepackage{subcaption}

% Verlinkt Textstellen im PDF-Dokument (sollte am Ende geladen werden)
\usepackage[unicode]{hyperref}
% „Schlaue“ Referenzen (nach hyperref laden!)
\usepackage{cleveref}

% --- Einstellungen
% -- LaTeX/KOMA
% 1,5-facher Zeilenabstand
\onehalfspacing
\recalctypearea
% Schrift bei Bildunterschriften ändern
\addtokomafont{caption}{\small}
\addtokomafont{captionlabel}{\bfseries}
% Nummerierung der Formeln entsprechend des Abschnitts (z.B. 1.1)
\numberwithin{equation}{section}
% „Verwaiste“ Zeilen am Seitenanfang/-Ende stärker vermeiden
\clubpenalty=1000
\widowpenalty=1000
% Auf mehrere Seiten aufgespaltene Fußnoten stärker vermeiden
\interfootnotelinepenalty=3000

% -- csquotes
% Anführungszeichen automatisch umwandeln
\MakeOuterQuote{"}

% -- siunitx
\sisetup{
	locale=DE,
	separate-uncertainty,
	output-product=\cdot,
	quotient-mode=fraction,
	per-mode=fraction,
	fraction-function=\sfrac
}

% -- hyperref
\hypersetup{
	% Links/Verweise mit Kasten der Dicke 0.5pt versehen
	pdfborder={0 0 0.5}
}

% -- cleveref
\crefname{equation}{}{}
\Crefname{equation}{}{}

% -- biblatex (Literaturverzeichnis)
\IfFileExists{res/literatur.bib}{
	\addbibresource{res/literatur.bib}
}{}

\AtEndPreamble{
	% Kopf- und Fußzeile konfigurieren
	\ifthenelse{\boolean{showHeader}}{
		\KOMAoptions{headsepline}
		\recalctypearea
		\automark{section}
		% Innenseite der Kopfzeile
		\ihead{\headmark}
		% Mitte der Kopfzeile
		\chead{}
		% Außenseite der Kopfzeile
		\ohead{\usekomafont{pagehead}\varAutor}
	}{}
	% Innnenseite der Fußzeile
	\ifoot{}
	% Mitte der Fußzeile          
	\cfoot{-~\pagemark~-}
	% Außenseite der Fußzeile
	\ofoot{}

	% Metadaten für die PDF-Datei
	\hypersetup{
		pdftitle={Versuchsprotokoll: \varName},
		pdfauthor={\varAutor},
		pdfsubject={F-Praktikum},
		pdfkeywords={Physik, Münster, Praktikum, Versuchsprotokoll}
	}
}

