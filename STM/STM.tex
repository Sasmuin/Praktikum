% Autor: Simon May
% Datum: 2017-10-05
% Diese Datei bietet ein minimalistisches Grundgerüst für ein LaTeX-Dokument,
% z.B. für die Bearbeitung der Aufgaben.
\documentclass[
	% Papierformat
	a4paper,
	% Schriftgröße (beliebige Größen mit „fontsize=Xpt“)
	12pt,
	% Schreibt die Papiergröße korrekt ins Ausgabedokument
	pagesize,
	% Sprache für z.B. Babel
	ngerman
]{scrartcl}

% Achtung: Die Reihenfolge der Pakete kann (leider) wichtig sein!
% Insbesondere sollten (so wie hier) babel, fontenc und inputenc (in dieser
% Reihenfolge) als Erstes und hyperref und cleveref (Reihenfolge auch hier
% beachten) als Letztes geladen werden!

% Silbentrennung etc.; Sprache wird durch Option bei \documentclass festgelegt
\usepackage{babel}
% Verwendung der Zeichentabelle T1 (Sonderzeichen etc.)
\usepackage[T1]{fontenc}
% Legt die Zeichenkodierung der Eingabedatei fest, z.B. UTF-8
\usepackage[utf8]{inputenc}
% Schriftart
\usepackage{lmodern}
% Zusätzliche Sonderzeichen
\usepackage{textcomp}

% Mathepaket (intlimits: Grenzen über/unter Integralzeichen)
\usepackage[intlimits]{amsmath}
% Ermöglicht die Nutzung von \SI{Zahl}{Einheit} u.a.
\usepackage{siunitx}
% Zum flexiblen Einbinden von Grafiken (\includegraphics)
\usepackage{graphicx}
% Abbildungen im Fließtext
\usepackage{wrapfig}
% Abbildungen nebeneinander (subfigure, subtable)
\usepackage{subcaption}
% Funktionen für Anführungszeichen
\usepackage{csquotes}
% Zitieren, Bibliographie
\usepackage{biblatex}

% Verlinkt Textstellen im PDF-Dokument
\usepackage[unicode]{hyperref}
% "Schlaue" Referenzen (nach hyperref laden!)
\usepackage{cleveref}

% siunitx: Deutsche Ausgabe, Messfehler getrennt mit ± ausgeben
\sisetup{
	locale=DE,
	separate-uncertainty
}

\begin{document}
\begin{titlepage}
	\centering
	{\scshape\LARGE Versuchsbericht zu \par}
	\vspace{1cm}
	{\scshape\huge STM - Rastertunnelmikroskop \par}
	\vspace{2.5cm}
	{\LARGE Gruppe 6 Mo\par}
	\vspace{0.5cm}
	{\large Nils Kulawiak (E-Mail: n\_kula01@wwu.de) \par}
	{\large Oliver Brune (E-Mail: o\_brun02@wwu.de) \par}
	{\large Anthony Pietz (E-Mail: a\_piet09@wwu.de) \par}
	\vfill
	durchgeführt am 17.12.2018\par
	
	\vfill
	betreut von Christoph Angrick
	{\large \today\par}
\end{titlepage}

\tableofcontents
		
\newpage
\section{Kurzfassung}
der Versuch beschäftigt sich mit dem Rastertunnelmikroskop. Im ersten Teil wird dabei die Theorie des STMs erläutert. Im zweiten Teil geht es darum eine Kupfer- und eine Goldprobe zu untersuchen. Dabei wird versucht die atomare Struktur beider Proben näher zu untersuchen. Für die Kupferprobe ist es möglich die einzelnen Atome zu sehen und auch die Gitterkonstante konnte mit $a = \SI{1,419 \pm 0,008} {\AA}$ bestimmt werden. Bei der Goldprobe allerdings gibt es das Problem, dass die freien Elektronen an der Oberfläche sehr gleichmäßig verteilt sind, weshalb nur einzelne Stufen auf der Goldoberfläche sichtbar werden. 

\section{Theorie}
In der Physik spielen Oberflächen für die Eigenschaften von Stoffen eine große Rolle. Eine Methode, um die Beschaffenheit von Oberflächen genauer zu untersuchen, ist die Rastertunnelmikroskopie (eng. scanning tunneling mikroskopy, kurz: STM). Die STM gehört zu den Techniken der Rastersondenmikroskopie und basiert auf dem  quantenmechanischen Tunneleffekt. Dieser ermöglicht es Elektronen eine Potentialbarriere zwischen zwei Leitern zu durchdringen, obwohl die Barriere höher ist als die Energie des Elektrons. Im STM entspricht die Potentialbarriere dem Abstand zwischen der Spitze des Mikroskops und der zu untersuchenden Probe. Wird eine Spannung zwischen Spitze und Probe angelegt, fließt daher ein kleiner Tunnelstrom. Der einfachste Fall ist der einer rechteckigen Potentialbarriere, durch die ein freies Elektron tunnelt. Die Tunnelwahrscheinlichkeit ist für diesen Fall durch

\begin{equation}
	T(E) \approx \exp [-\frac{2}{\hbar}\sqrt{2m(\phi - E)}s]
\end{equation}

gegeben, wobei $s$ die Dicke der Potentialbarriere, $E$ die Energie des Elektrons und $\phi$ die Höhe der Potentialbarriere darstellt. Diese Gleichung lässt sich auf den allgemeinen Fall einer beliebig geformten Potentialbarriere verallgemeinern. Dann ergibt eine WKB-Näherung für den Fall, dass das Potential im Vergleich zur Wellenlänge des Elektrons langsam variiert, folgende Gleichung:

\begin{equation}
	T(E) = \exp [-\frac{2}{\hbar} \int_{z_1}^{z_2}\sqrt{2m(\phi(z) - E)}dz].
\end{equation}

Im Fall der STM tunnelt allerdings kein freies Elektron durch eine Barriere, sondern die Elektronen sind im Metall gebunden und Tunnel von einem Metall ins andere. Das dazugehörige Potential ist in \cref{barrier} gezeigt. Deshalb werden nicht mehr die Energien der freien Elektronen, sondern die Fermi-Niveaus ($E_F$) der Spitze und der Probe betrachtet. Diese sind gegeneinander verschoben, da zwischen Spitze und Probe eine Spannung angelegt wird. Ist diese positiv, liegt das Fermi-Potential der Spitze höher als das der Probe, daher tunneln die Elektronen von der Spitze zur Probe. Bei negativer Spannung ist es andersherum. Außerdem ist die Potentialbarriere abhängig von den Austrittsarbeiten der Metalle von Probe und Spitze, die in der Regel unterschiedlich sind. Zuletzt besteht auch eine Abhängigkeit von an Probe und Spitze entstehenden Bildladungen.

\begin{figure}[h!]
	\centering
	\includegraphics[scale = 1.2]{barrier.png}
	\caption{Die Grafik zeigt die Potentialbarriere zwischen Spitze und Probe im STM. Aufgrund der Tunnelspannung sind die beiden Fermi-Niveaus gegeneinander verschoben. Da die angelegte Spannung hier positiv ist, fällt die Potentialbarriere zwischen Spitze und Probe linear ab. Für die Abrundung des Potentials an den Rändern sind Bildladungen verantwortlich.}
	\label{barrier}
\end{figure}

Um nun die Oberfläche des zu untersuchenden Stoffes abzubilden, wird die Spitze des STM in $x$- und $y$-Richtung über die Probe gefahren. Dabei wird für jede Einstellung der Tunnelstrom gemessen. Es gibt zwei verschiedene Messmodi: Einerseits ist es möglich, den Abstand zur Probe konstant zu halten, dann wird aus der Änderung des Tunnelstroms die Beschaffenheit der Oberfläche bestimmt. Zum anderen kann auch der Tunnelstrom konstant gehalten werden, indem mithilfe eines Regelkreises der Probenabstand entsprechend variiert wird. In diesem Versuch wurde die zweite Möglichkeit verwendet. Das STM erstellt somit ein Höhenprofil der Oberfläche. Die Höhe der Oberfläche entspricht dabei der lokalen Elektronenzustandsdichte. Dieses Profil kann atomgenau erstellt werden, da der Tunneleffekt auf sehr kleinen Abständen stattfindet und er bereits von Abstandsänderungen von weniger als $\SI{1}{\angstrom}$ beeinflusst wird. Daher ist es notwendig, dass die Spitze des Mikroskops sehr fein ist, im Optimalfall besteht sie nur aus einem einzelnen Atom, von dem die Elektronen in die Probe tunneln können. Außerdem wird für die Bestimmung des Höhenprofils der Tunnelstrom benötigt. Hierfür ist ein Modell nötig, das im folgenden eingeführt wird.

\subsection{Tersoff-Hamann-Modell}
Bei einer vollständigen dreidimensionalen Berechnung des Tunnelstroms ergibt sich eine komplizierte, analytisch nicht lösbare Gleichung, sodass nur mit hohem Rechenaufwand ein Ergebnis bestimmt werden könnte. Zur Vereinfachung stellten J. Tersoff und D.R. Hamann im Jahr 1985 ein vereinfachtes Modell vor, mit dem der Tunnelstrom genähert werden kann. Dieses Modell ist nach ihren Erfindern als Tersoff-Hamann-Modell bekannt. Hierbei wird der Tunnelstrom mithilfe der zeitabhängigen Störungstheorie erster Ordnung berechnet. Das Modell geht davon aus, dass sich an der Spitze des Mikroskops nur ein einzelnes Atom mit sphärischer Oberfläche befindet ,und dass die Probe eben ist (siehe \cref{spitze}).

\begin{figure}[h!]
	\centering
	\includegraphics[scale=0.8]{spitze.png}
	\caption{Der Tunnelkontakt nach dem Tersoff-Hamann-Modell. Die Annahmen des einzelnen Atoms an der Spitze mit Krümmungsradius $R$ und der ebenen Probe sind hier dargestellt.}
	\label{spitze}
\end{figure}

In der ersten Ordnung Störungstheorie, die hier verwendet wird, ergibt sich für den Tunnelstrom
\begin{equation}
	I = \frac{2\pi e}{\hbar} \sum_{\mu, \nu} |M_{\mu,\nu}|^2 f(E_\mu)[1-f(E_\nu + eV_b)] \delta(E_\mu - E_\nu),
	\label{eq:I1}
\end{equation}
wobei $E_\mu$ und $E_\nu$ die Energien der Zustände der Spitze bzw. Probe und $f$ die Fermi-Dirac-Verteilungsfunktion für die Zustände der Spitze ($\mu$) und der Probe ($\nu$) bezeichnet. $V_\text{b}$ ist die angelegte Spannung zwischen Spitze und Probe. Die Delta-Funktion sorgt dafür, dass nur elastische Tunnelprozesse betrachtet werden. $M_{\mu,\nu}$ ist das Matrixelement, dass den Übergang zwischen den Zuständen von Spitze und Probe beschreibt. Diese Gleichung lässt sich mit einigen simplen Annahmen deutlich vereinfachen. Nimmt man niedrige Temperatur und kleine Tunnelspannungen an, erhält man
\begin{equation}
	I = \frac{2\pi e^2}{\hbar}V_\text{b} \sum_{\mu, \nu} |M_{\mu,\nu}|^2 \delta(E_\mu - E_\text{F}) \delta(E_\nu - E_\text{F}),
	\label{eq:I2}
\end{equation}
mit der Energie des Fermi-Niveaus $E_\text{F}$.

Eine Möglichkeit zur Bestimmung des Matrixelements entwickelte John Bardeen im Jahr 1961. \cite{1} Die von ihm entdeckte Gleichung lautet

\begin{equation}
	M_{\mu,\nu} = \frac{\hbar^2}{2m_\text{e}} \int_{\text{d}S} \text{d}S(\Psi_\mu^* \nabla \Psi_\nu - \Psi_\nu \nabla \Psi_\mu^*).
	\label{eq:M}
\end{equation}
$\Psi_\mu$ und $\Psi_\nu$ sind die Zustände der Spitze bzw. der Probe. Der Ausdruck in der Klammer entspricht dem quantenmechanischen Stromdichteoperator. Integriert wird über $S$, dies ist eine Querschnittsfläche, die vollständig im Bereich zwischen den Elektroden liegt, also Spitze und Probe voneinander trennt. Um das Übergangsmatrixelement zu berechnen, müssen nun noch $\Psi_\mu$ und $\Psi_\nu$ bestimmt werden. Die Wellenfunktion der Spitze wird durch eine s-Wellenfunktion der Form

\begin{equation}
	\Psi_\mu = \frac{1}{\sqrt{V_\text{S}}} R \exp(\kappa R) \frac{1}{|r-r_0|} \exp(-\kappa|r-r_0|)
	\label{eq:psimu}
\end{equation}

beschrieben, wobei $V_\text{s}$ das Volumen der Spitze, $R$ den Krümmungsradius, $r_0$ die Position des Mittelpunktes des Spitzenendes, $\kappa = \frac{\sqrt{2m_\text{e}\Phi}}{\hbar}$ die inverse Abklinglänge der Welle im Vakuum und $\Phi$ die Austrittsarbeit beschreibt. Die Wellenfunktion der Probe ist durch

\begin{equation}
	\Psi_\nu = \frac{1}{\sqrt{V\text{P}}} \sum_{G} a_G \exp(-\sqrt{\kappa^2 + |k_\parallel + G|^2}z) \exp[i(G + k_\parallel)]
	\label{eq:psinu}
\end{equation}

gegeben mit Volumen der Probe $V_\text{P}$, einem reziproken Gittervektor $G$ mit Entwicklungskoeffizient $a_G$ und dem zur Oberfläche parallelen Teil des Bloch-Wellenvektor $k_\parallel$.
Zur Vereinfachung werden die Austrittsarbeiten von Spitze und Probe als gleich angenommen. Indem \cref{eq:psimu} und \cref{eq:psinu} in \cref{eq:M} und diese anschließend in \cref{eq:I2} eingesetzt werden, erhält man als Endergebnis

\begin{equation}
	I = \frac{32 \pi^3 e^2}{\hbar} V_{\text{b}} \frac{\Phi^2}{\kappa^4} \rho_\text{S}(E_\text{F}) R^2 \exp(2 \kappa R) \rho_\text{P}(r_0, E_\text{F}).
	\label{eq:I3}
\end{equation}

Hierbei sind $\rho_\text{S}(E_\text{F})$ die normierte Zustandsdichte der Spitze und $\rho\text{P}(r_0, E_\text{F}) = \sum_{\nu} |\Psi_\nu(r_0)|^2 \delta(E_\nu - E_\text{F})$ die lokale Zustandsdichte der Probe.

Aus \cref{eq:I3} ergeben sich einige wichtige Eigenschaften des Tunnelstroms im STM. Der Tunnelstrom hängt exponentiell vom Abstand zwischen Spitze und Probe ab.
Wird der Tunnelstrom während der Messung konstant gehalten, variiert die Höhe der Spitze je nach elektronischer Struktur der Probe an dieser Stelle. Diese Abhängigkeit von der lokalen Zustandsdichte am Fermi-Niveau ermöglicht es, auf diese Weise die Topographie der Probe zu kartographieren. Diese enthält Informationen darüber, um welchen Stoff es sich handelt und wie die Atome im Gitter sortiert sind. Außerdem können Korngrenzen und Defekte im Kristallgitter identifiziert werden.

Um ein möglichst genaues Ergebnis zu erhalten, muss das Auflösungsvermögen maximiert werden. Dies wurde abgeschätzt auf

\begin{equation}
	2\rho = 1,66 \sqrt{\frac{R + s}{\kappa}}
	\label{eq:sigma}
\end{equation}

Um eine möglichst gute Messung durchzuführen, sollten also an der Spitze des Rastertunnelmikroskops nur ein einzelnes Atom sitzen. Außerdem sollte der Abstand zwischen Spitze und Probe so klein wie möglich sein.

Mit diesem Modell lässt sich die Theorie hinter dem Rastertunnelmikroskop bereits gut erklären, gleichzeitig ist es aber auch noch mit relativ einfachen Mitteln berechenbar. Für eine genauere Berechnung wären weitere Informationen über die Wellenfunktion der Spitze nötig. Außerdem ließe sich ein komplexeres Modell nicht mehr auf eine einfache Formel reduzieren.

\newpage


\section{Methode}
Der wichtigste Teil des STMs ist der Messkopf. In diesem wird die Oberfläche der Probe abgetastet, indem der Tunnelstrom, der zwischen Messspitze und Probe fließt, mithilfe eines Operationsverstärkers in eine Spannung umgewandelt wird. Diese Spannung stellt das Messsignal dar. Damit Atome messbar werden muss die Messspitze auf ein $\AA$ genau eingestellt werden. 

Dafür ist der Messkopf auf einer Steinplatte mit einer Gummimatte obendrauf gelagert, um Schwingungen zu minimieren. Die schwere Steinplatte steht wiederum auf 4 kleineren Gummifüßen. 
Das System funktioniert so, dass die Eigenschwingung zwischen Steinplatte und Gummifüßen bei sehr kleinen Frequenzen stattfinden. Außerdem haben die Gummifüße eine große Reibung und der Übertrag von Steinplatte auf Gummimatte ist wiederum sehr gering. Das führt dazu, dass leichte Schwingungen des Tisches praktisch keine Auswirkung auf die Messung haben.

Die Messspitze ist ein Stück eines Pt-Ir Drahtes mit einer Länge von 5-10 mm und eine Breite von 0,1 mm. Die Spitze ist durch eine Blattfeder in der Kerbe einer Keramikplatte festgehalten. Eins der drei Piezoelemente ist direkt an der Keramikplatte befestigt. Ein Piezoelement ist ein Kristall an dem durch Anlegung einer Spannung eine sehr kleine mechanische Kraft erzeugt werden kann, welche zur Positionierung der Spitze verwendet wird. Es werden drei verschiedene Piezoelemente gebraucht, um die Spitze in alle drei Raumrichtungen mit sehr hoher Genauigkeit zu bewegen.

Zur Durchführung muss zunächst die Messspitze selbst hergestellt werden. Dafür wird ein Pt-Ir Draht abgeschnitten und mit einer Flachzange festgehalten. Gleichzeitig muss der Seitenschneider im 20°-45° Winkel zum Draht gehalten werden. Daraufhin muss der Seitenschneider und die Flachzange auseinandergezogen werden, während gleichzeitig der Draht mit dem Seitenschneider geschnitten wird. Im Optimalfall ersteht dabei eine Messspitze mit einer einatomigen Spitze, weil sonst die Qualität der Bilder stark leidet. Außerdem muss darauf geachtet werden, dass alle Geräte sauber sind, da schon leichte Verunreinigungen zu schlechteren Ergebnissen führen kann. Zudem muss die Gold- bzw. Graphitprobe eingesetzt werden, wobei besonders bei der Goldprobe aufgepasst werden muss, weil sich diese nicht reinigen lässt.


Sobald sowohl die Messspitze als auch die Probe eingesetzt wurden wird Messspitze zuerst grob per Hand und dann später mit den eben genannten Piezoelementen auf 1 $\AA$ an die Probe herangeführt. 

Der Rest der Messung lässt sich mit dem Computer bzw. dem Easy-Scan Programm durchführen. Es muss sichergestellt werden, dass die Oberfläche der Probe senkrecht zur Messspitze steht. Danach müssen nur noch Bilder der Proben unter verschiedenen Parametern gemacht werden. 
Alle Messungen wurden mit den Werten: SetPoint(Tunnelstrom): 1,001nA; P-Gain: 12;
I-Gain: 11; GapVoltage: 0,050V; durchgeführt.
\section{Auswertung}
\subsection{Graphit}
Zuerst wird die Graphitprobe untersucht. Dafür muss allerdings erst klargestellt werden, wie das Graphitgitter aufgebaut ist. Ein Modell des Graphitgitter ist in \cref{gra1} zu sehen, dabei wird klar, dass die Ebenen jeweils zueinander versetzt sind, was dazu führt, dass einzelne Atome besser zu sehen sind als andere. Diese Atome können sich in drei verschiedene Gruppen aufteilen lassen:A,B und H, wie es in \cref{gra2} explizit zu sehen ist. Die oberen Atome sind dabei als weißer Kreis aufgezeichnet und die unteren als schwarzer Punkt. Somit setzten sich die drei Gruppen zusammen. A-Atome haben einen direkten Nachbarn in der Netzebene unter ihnen. B-Atome hingegen fehlt dieser Nachbar, während die H-Atome nur ein Atom in der zweiten Netzebene haben. Weil das Rastertunnelmikroskop den Ladungsstrom misst und dieser wiederum direkt von der Ladungsdichte abhängt verhalten sich die verschiedenen Atome auch unterschiedlich. Die A-Atome sind die durch die weiteren Atome direkt unter ihnen stärker gebunden, was dazu führt dass sich die Ladungsverteilung tiefer in dem Material befindet. Das macht sie für das Rastertunnelmikroskop praktisch unsichtbar.

\begin{figure}[h!]
	\centering
	\includegraphics[scale=0.7]{graphitgitter1.png}
	\caption{seitlich gezeigtes Modell eines Graphitgitters}
	\label{gra1}
\end{figure}


\begin{figure}[h!]
	\centering
	\includegraphics[scale=0.7]{graphitgitter2.png}
	\caption{Modell eines Graphitgitters von oben}
	\label{gra2}
\end{figure}

Mit diesem Wissen ist es allerdings möglich das Graphit zu untersuchen und auch die Gitterkonstante zu berechnen. Um die Gitterkonstante zu berechnen wurden verschiedene Bilder wie beispielsweise \cref{scan3} benutzt. Dafür wird jeweils der Abstand von mehreren Gitterabständen genommen und dieser dann gemittelt. Dieser Vorgang wird mehrfach wiederholt und diese Werte wiederum gemittelt. Damit folgt eine mittlere Gitterkonstante der B-Atome von $a_{B} = \SI{2,457 \pm 0,012}{\AA}$. Die Unsicherheit folgt dabei aus der Größe der einzelnen B-Atome und der Mittelwertabweichung. Die eigentliche Gitterkonstante kann mithilfe des Kosinussatzes berechnet werden. Da es sich jeweils um ein gleichschenkliges Dreieck handelt (vgl. \cref{gra2}) kürzt sich dieser zu
\begin{equation}
a^{2} = a_{B}^2(1 - 2 \cos(\theta)),
\end{equation}
wobei $\theta$ dem Winkel zwischen A- und B-Atomen entspricht. Für das im Modell angenommene Hexagon beträgt der Winkel $\theta$ genau 120°. Für ein einzelne Messung können diese 120° nicht angenommen werden, da das Bild leicht verzerrt ist. Da allerdings in alle möglichen Richtung Messungen betätigt wurden, kann diese Annahme zumindest näherungsweise getroffen werden. Daraus folgt dann eine Gitterkonstante von $a = \SI{1,419 \pm 0,008} {\AA}$.

Die gerade eben genannte Verzerrung kommen im Normalfall durch sogenannte Drifts zustande, was Temperaturänderungen entspricht. Diese sind sehr schwer komplett einzudämmen, da schon Temperaturänderungen von $\dfrac{1}{10}$°C zu deutlichen Veränderungen führen. In diesem Fall ist das Bild in y-Richtung gestreckt, wie es auch in \cref{scan1} noch einmal deutlich wird.  


\begin{figure}[h!]
	\centering
	\includegraphics[scale=0.4]{Scan.png}
	\caption{Messung der Graphitprobe mit dem STM auf 3nm}
	\label{scan3}
\end{figure}

\begin{figure}[h!]
	\centering
	\includegraphics[scale=0.4]{Scan1.png}
	\caption{Messung der Graphitprobe mit dem STM auf 1nm}
	\label{scan1}
\end{figure}
\newpage
\section{Gold}
Gold ist wesentlich schwieriger zu untersuchen als Graphit, da die freien Elektronen sehr gleichmäßig verteilt sind. Einzelne Atome sind deshalb selten bis gar nicht zu erkennen. Allerdings ist es möglich einzelne Stufen zu erkennen, an denen die Oberfläche ein Atom größer bzw. kleiner wird. Außerdem ist es manchmal möglich einzelne Kristallfehler zu entdecken. 

\begin{figure}[h!]
	\centering
	\includegraphics[scale=0.6]{Stufe.png}
	\caption{Messung der Goldprobe}
	\label{gold}
\end{figure}


In \cref{gold} sind einige dieser Stellen zu sehen. Die einzelnen Stufen sind besonders gut im  Strombild(current) zu sehen, wobei jede Erhöhung eine atomare Stufe darstellt. Die schwarzen Stellen sind höchstwahrscheinlich Gitterfehler, an denen das STM nichts messen konnte.

\section{Diskussion}
Insgesamt konnten die wichtigsten Beobachtungen erfolgreich gemacht werden. Die Struktur von Graphit ist, soweit wie möglich, in \cref{gra1}, deutlich zu erkennen und auch die berechnete Gitterkonstante mit $a = \SI{1,419 \pm 0,008} {\AA}$ stimmt gut mit dem Literaturwert von $a = \SI{1,42} {\AA}$ überein. In Gold war wie zu erwarten die atomare Struktur in \cref{gold} nicht zu erkennen, allerdings ist es möglich mehrere Stufen und auch einige Gitterfehler zu sehen. 

Die größte Schwierigkeit des Versuchs war dabei das Herstellen einer funktionierenden Messspitze. Der aktuelle Verfahren ist, besonders für ungeübte Praktikanten, sehr schwierig und kann bei Versagen dazu führen, dass überhaupt keine sinnvollen Messwerte aufgenommen werden können. Möglicherweise wäre es sinnvoll eine vorgefertigte Messspitze bereitzustellen, um die Durchführung vom Rest des Versuches zu gewährleisten.
 

\section{Anmerkung}
Alle Messwerte sind vom Ordner BA-B-07, Ba-B-05 der Example-Data

\newpage
\begin{thebibliography}{9}
	\bibitem{A}
	J. Bardeen, Phys. Rev. Lett. 6, 57 (1961).
	
	\bibitem{B}
	J. Tersoff, D. R. Hamann, Phys. Rev. Lett. 50, 1998 (1983)
	
	\bibitem{C}
	\textit{Raster-Tunnel-Mikroskop, Spitzenpräparation und Messungen an Graphit und Gold },
	FD-Praktikum, (2010)

\end{thebibliography}
\end{document}