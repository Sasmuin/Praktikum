\section{Fourierfilterung}
\subsection{Ergebnis}
Zuerst soll ein "Fourier" Schriftzug vor einem Gitter modelliert werden, indem das Gitter entfernt wird. Dazu wird eine Tiefpassfilterung benutzt, da die im Vergleich große Schrift hauptsächlich aus niedrigen Frequenzen besteht. 


\begin{figure}[h]
\begin{subfigure}[c]{0.5\textwidth}

\includegraphics[width=0.9\textwidth]{Fourier.png}
	      \caption{}
          \label{fig:NiceImage1}
          
\end{subfigure}
\begin{subfigure}[c]{0.5\textwidth}
	\includegraphics[width=0.9\textwidth]{Fourier_Filter.png}
	      \caption{}
          \label{fig:NiceImage2}
\end{subfigure}
\caption{In \cref{fig:NiceImage1} ist der "Fourier" Schriftzug ohne Filter zusehen; in \cref{fig:NiceImage2} mit einem Tiefpassfilter.}
\label{Fourier}
\end{figure}   

In \cref{Fourier} ist zu sehen, wie sich der Tiefpassfilter auf das Bild auswirkt. Das Gitter ist zum größten Teil nicht mehr als solches zu erkennen.