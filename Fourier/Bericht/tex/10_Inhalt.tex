\section{Abstract}
Dieser Versuch beschäftigt sich mit der optischen Fouriertransformation eines Laserstrahls an einem Objekt und den Möglichkeiten, die sich daraus ergeben. Zuerst wird das Interferenzbild eines Gitters in unterschiedlichen Entfernungen betrachtet, um so den Übergang von der Fresnel- zur Fraunhoferbeugung darzustellen. Anschließend werden die Gitterkonstanten fünf verschiedener Gitter mithilfe des Abstands der Beugungsmaxima im Fernfeld bestimmt.



\section{Theorie\cite{anleitung-ws2014}}
\subsection{Fouriertransformation}
Die Fouriertransformation wird in der Signalverarbeitung verwendet, um Signale in den Frequenzraum und zurück zu transformieren. Dabei können sowohl zeitliche Signale in ihre zeitlichen Frequenzen zerlegt werden, als auch räumliche Signale in ihre räumlichen Frequenzen. \cref{eq:tfourier} zeigt die Gleichung für die Umwandlung zeitlicher Signale, \cref{eq:xyfourier} zeigt die Gleichung für die Umwandlung zweidimensionaler räumlicher Signale.

\begin{equation}
	\mathcal{F}\left[ f\left( t\right) \right] \left( \nu\right) = \frac{1}{\sqrt{2\pi}} \int_{-\infty}^{\infty} f\left( t\right) e^{-2\pi i \nu t} \text{d}t = F\left( \nu\right) 
	\label{eq:tfourier}
\end{equation}

\begin{equation}
	\mathcal{F}\left[ f\left( x,y\right) \right] \left( \nu_x, \nu_y\right) = \frac{1}{\sqrt{2\pi}} \int_{-\infty}^{\infty} \int_{-\infty}^{\infty} f\left( x, y\right) e^{-2\pi i (\nu_x x + \nu_y y)} \text{d}x \text{d}y = F\left( \nu_x, \nu_y\right) 
	\label{eq:xyfourier}
\end{equation}

Dabei sind $F\left( \nu\right)$ bzw. $F\left( \nu_x, \nu_y\right)$ die Funktion im Frequenzraum, $f\left( t\right)$ die Funktion im Zeitraum, $f\left( x,y\right)$ die Funktion im Ortsraum, $t$ die Zeit, $x$ und $y$ die räumlichen Koordinaten, $\nu$ die zeitliche Frequenz und $\nu_x$ und $\nu_y$ die räumlichen Frequenzen. Die jeweiligen Rücktransformationen werden durch \cref{eq:trück}(zeitlich) und \cref{eq:xyrück}(räumlich) beschrieben.

\begin{equation}
	\mathcal{F}^{-1}\left[ f\left( \nu\right) \right] \left( t\right) = \frac{1}{\sqrt{2\pi}} \int_{-\infty}^{\infty} F\left( \nu\right) e^{-2\pi i \nu t} \text{d}\nu = f\left( t\right) 
	\label{eq:trück}
\end{equation}

\begin{equation}
	\mathcal{F}^{-1}\left[ f\left( \nu_x,\nu_y\right) \right] \left( x, y\right) = \frac{1}{\sqrt{2\pi}} \int_{-\infty}^{\infty} \int_{-\infty}^{\infty} F\left( \nu_x,\nu_y\right) e^{-2\pi i (\nu_x x + \nu_y y)} \text{d}\nu_x \text{d}\nu_y = f\left( x, y\right) 
	\label{eq:xyrück}
\end{equation}

Ein Beispiel für die Fouriertransformation ist in \cref{rechteckpuls} dargestellt. Dort ist ein Rechteckpuls und seine Fouriertransformierte, eine sinc-Funktion, in einer Dimension dargestellt. 

\begin{figure}[h!]
	\centering
	\includegraphics[width=0.9\textwidth]{rechteckpuls.png}
	\caption{Fouriertransformation, angewandt auf einen Rechteckpuls. Das Ergebnis ist eine sinc-Funktion. Entnommen aus \cite[3]{anleitung-ws2014}}
	\label{rechteckpuls}
\end{figure}

Bei der optischen Fouriertransformation ist allerdings eine Transformation in zwei räumlichen Dimensionen nötig. Dies ist in \cref{rechteckspalt} dargestellt. Hier wird ein Rechteckspalt fouriertransformiert, heraus kommen eine sinc-Funktion in x- und eine sinc-Funktion in y-Richtung, die sich überlagern.

\begin{figure}[h!]
	\centering
	\includegraphics[width=0.7\textwidth]{rechteckspalt.png}
	\caption{Ein Rechteckspalt und die Fouriertransformierte, zwei sich überlagernde sinc-Funktionen. Entnommen aus \cite[4]{anleitung-ws2014}}
	\label{rechteckspalt}
\end{figure}

Die Fouriertransformation ist außerdem ein nützliches Werkzeug, um die Faltung zweier Funktionen zu beschreiben. Diese wird häufig verwendet, um periodische Strukturen wie Gitter zu beschreiben. Die Faltung ist im Orts- bzw. Zeitraum ein kompliziert zu bestimmendes Integral. Allerdings ist die Faltung zweier Funktionen in einem dieser Räume einfach gleich dem Produkt der Funktionen im Frequenzraum.

\subsection{Skalare Beugungstheorie}
Um die Eigenschaften des Lichts in diesem Versuch zu beschreiben, wird die skalare Beugungstheorie verwendet. Diese hat ihren Namen von der Näherung, nach der das elektrische Feld nicht als Vektor, sondern als skalare, monochromatische ebene Welle beschrieben wird. Trifft das Licht auf ein Hindernis, wird es gemäß dem Huygenschen Prinzip gebeugt. Dies besagt, dass jeder Punkt einer Wellenfront wieder als Ausgangspunkt einer Kugelwelle dient. Diese Kugelwellen interferieren nun miteinander, sodass auf einem Schirm, der hinter dem Hindernis aufgestellt wird, ein Beugungsbild beobachtet werden kann. Die Beugungsbilder, die bei verschiedenen Abständen auftreten, sind in \cref{Theoriebild} dargestellt. Dabei ist $z_0$ der Abstand des Beugungsbilds vom Hindernis und $b$ die Größe des Beugungsobjekts.

\begin{figure}[h!]
	\centering
	\includegraphics[width=0.9\textwidth]{abstande.png}
	\caption{Fresnel- und Frauenhofer Näherung. Im Fernfeld kann die Frauenhofer Näherung angewandt werden, wohingegen im Nahfeld die Fresnel Näherung angewandt wird. Entnommen aus \cite[5]{anleitung-ws2014}}
	\label{Theoriebild}
\end{figure}

In der Nahzone wird die Lichtausbreitung mithilfe der Fresnel-Näherung beschrieben. Diese gilt, wenn der Abstand des Beugungsobjekts zum Schirm klein ist. Ist dieser Abstand hingegen groß, also gilt $z_0 >> b$, wird das Beugungsbild mithilfe der Fraunhofer-Näherung für das Fernfeld beschrieben. Das elektrische Feld in der Fraunhofer-Näherung wird lässt sich mit \cref{eq:fraunhofer} bestimmen.

\begin{equation}
E\left( x', y', z'\right) = A\left( x', y', z'\right) \mathcal{F}\left( E_0\left( x, y, z_0\right) e^{i\Phi\left( x, y, z_0\right) }\right) \left( \nu_x, \nu_y\right) 
\label{eq:fraunhofer}
\end{equation}

Das Interferenzbild im Fernfeld entspricht also der Fouriertransformierten des Beugungsobjekts.

\subsection{Beugung am Gitter}
Bei der Beugung am Gitter entstehen nach dem Huygenschen Prinzip Kugelwellen, die mit einander interferieren. Die Interferenz ist entweder positiv bei einer Phasendifferenz von $2\pi$ oder negativ bei einer Phasendifferenz von $\pi$. Es entsteht also ein Beugungsbild, das vom Abstand zwischen Gitter und Schirm $d$, der Gitterkonstante $b$, der Spaltzahl $N$ und dem Beugungswinkel $\theta$ abhängig ist. Dies kann mit \cref{eq:beugung} berechnet werden.

\begin{equation}
I(\theta) = I_0 \text{sinc}^2\left( \frac{N\pi d \sin(\theta)}{\lambda}\right)  \text{sinc}^2\left( \frac{\pi b \sin(\theta)}{\lambda}\right) 
\label{eq:beugung}
\end{equation}

Um den Gitterabstand zu bestimmen, benötigt man allerdings nur die Bedingung für positive Interferenz. Sie lautet:

\begin{equation}
	k \lambda = b \sin(\theta); k\in \mathbb{N}_0
\end{equation}

\subsection{Optische Fouriertransformation mit Linsen}
In diesem Versuch soll hauptsächlich im Bereich der Fraunhofer-Näherung gearbeitet werden. Diese gilt aber erst im unendlichen. Daher wird eine Linse im Strahlengang positioniert, die das Bild der Fouriertransformierten in der Brennebene darstellt. Dieser Aufbau wird als 2f-Aufbau bezeichnet, da seine Länge der zweifachen Brennweite der Linse entspricht. Der Aufbau kann um eine weitere Linse erweitert werden, die eine Rücktransformation ausführt, um so wieder das ursprüngliche Bild zu erhalten. Die Linse wird dabei im Abstand $f$ von der Fourierebene positioniert. Im Abtand $f$ auf der anderen Seite der Linse wird dann das Objekt sichtbar. Der dazu nötige Aufbau ist in \cref{4f} dargestellt und wird 4f-Aufbau genannt.

\begin{figure}[h!]
	\centering
	\includegraphics[width=0.9\textwidth]{4f.png}
	\caption{Schematische Darstellung des 4f-Aufbaus, der hier zur Fourierfilterung eingesetzt wird. Entnommen aus \cite[6]{anleitung-ws2014}}
	\label{4f}
\end{figure}

\subsection{Frequenzfilterung}
Mithilfe des 4f-Aufbaus kann nun eine Fourierfilterung durchgeführt werden. Hierfür können verschiedene Filter in der Fourierebene platziert werden. Ein Tiefpass wird dabei mithilfe einer Blende realisiert, die nur Licht in der Mitte der nullten Beugungsordnung durchlässt. Ein Hochpass wird mit einer Nadel erzeugt, die mittig in der Fourierebene platziert wird. Außerdem können auch bestimmte Frequenzen gefiltert werden, um periodisches Rauschen zu entfernen.

\subsection{Dunkelfeldmethode}
Die Intensität eines Lichtfeldes, das Betragsquadrat des elektrischen Feldes kann auf einem Schirm mit bloßem Auge erkannt werden. Das elektrische Feld selbst ist allerdings nicht direkt sichtbar, da keine Informationen über die Phase des Felds bekannt sind. Um das elektrische Feld trotzdem zu detektieren, wird die Dunkelfeldmethode verwendet. Dazu nehmen wir an, dass das beugende Objekt dem elektrischen Feld eine ortsabhängige Phase aufprägt. Das E-Feld lässt sich nun schreiben als:

\begin{equation}
	E(x, y) = a e^{i\Phi\left( x, y\right) }.
\end{equation}

Für kleine Phasenänderungen gilt dann:

\begin{equation}
	E(x, y) = a\left( 1 + i\Phi\left( x, y\right) \right) .
\end{equation}

Fouriertransformiert ergibt sich dann:

\begin{equation}
	\mathcal{F}\left( E\left( x, y\right) \right)  = a\left( \delta\left( \nu_x\right) \delta\left( \nu_y\right) + \mathcal{F}\left( i\Phi\left( x, y\right) \right) \right) .
\end{equation}

Wird nun der erste Summand gefiltert, ergibt sich nach der Rücktransformation:

\begin{equation}
	\mathcal{F}^{-1}\left( \mathcal{F}\left( E\left( x, y\right) \right) \right) = E\left( x, y\right) = ia\Phi\left( x, y\right) .
\end{equation}

Die Intensität ist nun also abhängig von der Phase des elektrischen Feldes, somit ist die Phase auf dem Schirm sichtbar.

\section{Methoden}
Die optische Fouriertransformation wird in diesem Versuch mit einem Helium-Neon-Laser durchgeführt. Dieser strahlt entlang einer optischen Bank, auf der verschiedene optische Instrumente befestigt sind, auf eine Kamera. Das Bild der Kamera wird direkt auf einen Computer übertragen. Zur Fouriertransformation wird ausgenutzt, dass das Beugungsbild eines Objekts im Unendlichen die Fouriertransformation ist. Mithilfe einer Linse kann die Fouriertransformation in der Brennebene der Linse sichtbar gemacht werden. Dieser Aufbau wird auch als 2-f-Aufbau bezeichnet. Wird hinter der Brennebene der ersten Linse eine zweite Linse aufgestellt, wird die Fouriertransformation rückgängig gemacht und in der Brennebene der zweiten Linse wird das ursprüngliche Bild wieder sichtbar. Dies ist unter dem Namen 4-f-Aufbau bekannt. Wird nun in der Fourierebene ein Teil des Strahls geblockt, werden einzelne Frequenzen des Bildes gefiltert. Diesen Vorgang bezeichnet man als optische Fourierfilterung.

\section{Übergang von Nah- zu Fernfeld}
In diesem Versuchsteil wird das Beugungsbild eines Gitters in Abhängigkeit des Abstandes zwischen Schirm und Gitter beobachtet. Der schematische Aufbau, um diesen Sachverhalt zu untersuchen, ist in \cref{Lana1} zu sehen.
Dabei wird ein kollimierter Laserstrahl mit einer Wellenlänge von $\lambda = 633 nm$ auf einen Schirm gerichtet. 
\begin{figure}[h!]
	\centering
	\includegraphics[scale = 0.65]{Lana-Bild1.png}
	\caption{Aufbau zur visuellen Bestimmung des Nah-/Übergans- und Fernfeldes. Dies ist ein skizzenhafter Aufbau des Versuchs.}
	\label{Lana1}
\end{figure}
Der Laserstrahl wird, um ihn zu kollimieren, auf eine Linse gerichtet.
Danach trifft der Laserstrahl auf ein Gitter, welches in verschiedenen Abständen vom Laser, in den Strahlengang gebracht wird. Die Abstände zwischen Laser und Gitter variieren von 500mm bis 3800mm. Auf dem Schirm kann ein Beugungsbild erkannt werden, welches mittels eines Intensitätsmessgerätes aufgenommen wird.
Mit diesem Aufbau kann nun das Beugungsbild bei unterschiedlichem Abstand zwischen Schirm und Gitter beobachtet werden. \cref{alle} zeigt den Verlauf der Beugungsbilder bei veränderlichem Abstand. Je größer der Abstand vom Schirm zum Gitter wird, desto kleiner wird der Abstand zwischen Gitter und Laser. Je näher sich das Gitter zum Laser bewegt, desto mehr wird sich das Fernfeld einstellen.
\begin{figure}[h!]
	\centering
	\includegraphics[scale = 0.65]{alleabstande.png}
	\caption{Es sind die Beugungsbilder in Abhängigkeit der Abstände zwischen Schirm und Gitter aufgetragen. Oben Links ist das Beugungsbild im Nahfeld zu erkennen bei einem Abstand von 500mm. Unten rechts ist das Beugungsbild im Fernfeld zu erkennen bei einem Abstand von 3800mm. Von oben links nach unten rechts nimmt der Abstand zu. Der Abstand nimmt in 300mm Abständen zu.}
	\label{alle}
\end{figure}
In \cref{alle} kann in einem Abstand von $500 mm$ bis $1400 mm$ das Nahfeld erkannt werden. Bei einem Abstand von $1700 mm$ bis $2600 mm$  kann das Übergangsfeld beobachtet werden und in einem Abstand von $2900 mm$ bis $3800 mm$ kann das Fernfeld beobachtet werden. Die angegebenen Grenzen sind keine scharfen Grenzen. Häufig kann man an den Grenzen beide Effekte erkennen. Die Effekte sind beim Nah- und Fernfeld vor allem an den Rändern sichtbar (also bei $500 mm$ bzw. bei $3800 mm$) und beim Übergangsfeld bei einem Abstand von $2000 mm$. Diese Angaben wurden mit \cref{Theoriebild} erhoben. Gründe für diese Einteilung sind, dass sich die Maxima höherer Ordnung erst im Fernfeld klar unterscheiden lassen und sich die Maxima im Nahfeld in einem Punkt treffen. Charakteristisch für die Übergangszone ist das vermischen beider Effekte. Das Hauptmaxima ist deutlich zu erkennen, jedoch gibt es entfernt vom Hauptmaxima noch Helligkeitserscheinungen, die nach außen hin abschwächen und kaum zu erkennende Peaks ausbilden. Diese Charakteristiken können im Vergleich von \cref{Theoriebild} und \cref{alle} an den oben genannten Bildern beobachtet werden.

\section{Bestimmung der Gitterkonstanten}
In diesem Abschnitt wurden die Gitterkonstanten verschiedener Gitter bestimmt. Dafür wurde das Gitter in den Strahlengang gestellt. Die Kamera wird $\SI{3800}{mm}$ entfernt positioniert. Das Beugungsbild von Gitter 5 ist exemplarisch in \cref{Gitter5} dargestellt.

\begin{figure}
	\centering
	\includegraphics[scale=0.4]{Gitter5.jpg}
	\caption{Das Beugungsbild des 5. Gitters.}
	\label{Gitter5}
\end{figure}

Aus dem Abstand der einzelnen Beugungsmaxima zum nullten Beugungsmaximum $a$ und dem Abstand zwischen Gitter und Kamera $d$ kann dann mit \cref{eq:sin} $\theta$ bestimmt werden.

\begin{equation}
	\theta = \arctan\left(\frac{a}{d}\right)
	\label{eq:sin}
\end{equation}

Nun lässt sich die Gitterkonstante des untersuchten Gitters mit \cref{eq:gk} bestimmen, mit $k$ der Nummer des Beugungsmaximums.

\begin{equation}
	k\lambda = b \sin(\theta)
	\label{eq:gk}
\end{equation}

Bei jedem Gitter wurde die Gitterkonstante jeweils viermal bestimmt, mit den beiden ersten Beugungsmaxima nach oben und unten ausgehend vom nullten Maximum. Aus diesen wurde dann der Mittelwert gebildet. Die Ergebnisse sind in \cref{tab} zusammengefasst.

\begin{table}[h]
	\caption{Die Gitterkonstanten der fünf untersuchten Gitter in mm}
\begin{tabular}{lllll}
	Gitter 1 & Gitter 2& Gitter 3& Gitter 4& Gitter 5\\
	 $0,386\pm0,048$ & $0,456\pm0,053$ & $0,363\pm0,041$ & $0,323\pm0,032$ & $0,252\pm0,020$
\end{tabular}
\label{tab}
\end{table}

Die Unsicherheit beim Ablesen der Abstände wurde auf $\SI{\pm1}{mm}$ geschätzt. Daraus ergibt sich mithilfe der Formel für die Fehlerfortpflanzung (\cref{eq:utheta}) die Unsicherheit von $\theta$ und daraus mit \cref{eq:ugk} die Gesamtunsicherheit der Gitterkonstanten.

\begin{align}
u(\theta) &= \frac{u(a)*d}{a^2 +d^2}
\label{eq:utheta}\\
u(b) &= \frac{u(\theta) k \lambda \cos(\theta)}{sin(\theta)^2}
\label{eq:ugk}
\end{align}

\section{Fourierfilterung}
\subsection{Ergebnis}
\subsection{Fourier Schriftzug}
Zuerst soll ein "Fourier" Schriftzug vor einem Gitter modelliert werden, indem das Gitter entfernt wird. Dazu wird eine Tiefpassfilterung benutzt, da die im Vergleich große Schrift hauptsächlich aus niedrigen Frequenzen besteht. 


\begin{figure}[h]
\begin{subfigure}[c]{0.5\textwidth}

\includegraphics[width=0.9\textwidth]{Fourier.png}
	      \caption{}
          \label{fig:NiceImage1}
          
\end{subfigure}
\begin{subfigure}[c]{0.5\textwidth}
	\includegraphics[width=0.9\textwidth]{Fourier_Filter.png}
	      \caption{}
          \label{fig:NiceImage2}
\end{subfigure}
\caption{In \cref{fig:NiceImage1} ist der "Fourier" Schriftzug ohne Filter zu sehen; in \cref{fig:NiceImage2} mit einem Tiefpassfilter.}
\label{Fourier}
\end{figure}   

In \cref{Fourier} ist zu sehen, wie sich der Tiefpassfilter auf das Bild auswirkt. Das Gitter ist zum größten Teil nicht mehr als solches zu erkennen.

\subsection{Nenner eines Bruches entfernen}
Daraufhin soll der Nenner eines $\frac{1}{2}$ Bruchs entfernt werden. Wie in \cref{fig:Bruch} zu sehen ist, befindet sich unter dem Zähler ein Gitter, welches sich nicht bei dem Nenner befindet. Es werden also praktisch beide Zahlen entfernt, allerdings ist durch das Gitter immer noch der Umriss der 1 sichtbar. Dieses entfernen geschieht durch eine Hochpassfilterung, was bedeutet, dass niedrige Frequenzen entfernt werden. Das hat sehr gut funktioniert, da in \cref{fig:Bruch_filter} die Zwei, also der Nenner nicht mehr erkennbar ist.

\begin{figure}[h]
	\begin{subfigure}[c]{0.5\textwidth}
		
		\includegraphics[width=0.6\textwidth]{Bruch.png}
		\caption{}
		\label{fig:Bruch}
		
	\end{subfigure}
	\begin{subfigure}[c]{0.5\textwidth}
		\includegraphics[width=0.58\textwidth]{Filter_Bruch.png}
		\caption{}
		\label{fig:Bruch_filter}
	\end{subfigure}
	\caption{In \cref{fig:Bruch} ist der $\frac{1}{2}$ Bruch ohne Filter zu sehen; in \cref{fig:Bruch_filter} mit einem Hochpassfilter, wodurch die Zwei entfernt wurde.}
	\label{Bruch}
\end{figure}   

\subsection{Tiefpassfilterung bei einem Quadratgitter}
Als nächstes soll ausprobiert werden, was passiert, wenn ein Quadratgitter eingesetzt wird, das mit einem eindimensionalem Tiefpassfilter in verschiedenen Ausrichtungen gefiltert wird. Das ungefilterte Bild ist in \cref{fig:Gitter} zu sehen. Dazu im Vergleich wurde in \cref{fig:0Gitter} ein Tiefpassfilter im Winkel von 0° Grad eingebaut. Da der Tiefpass alle Frequenzen außer denen, die im 0° Winkel sind durchgelassen. Deshalb wurde erwartet, dass die Linien orthogonal zum Frequenzbild erkennbar sind. Allerdings ist das besonders am Rand und zum Teil auch in der Mitte des Bildes nicht immer deutlich zu erkennen. 

Ähnliche Probleme gibt es auch in \cref{fig:45Gitter} und \cref{fig:90Gitter}, in denen der Tiefpassfilter jeweils um 45° und 90° gedreht sind. Besonders in \cref{fig:45Gitter} lässt sich die Ausrichtung nur noch erahnen, während bei der 90° Drehung das Muster nur in der Mitte des Bildes unterbrochen wird.

Diese hellen Strahlen, die sich an der räumlichen Ausrichtung des Filters orientieren und damit die erwarteten Muster unterbrechen, stammen höchstwahrscheinlich daher, dass der Tiefpassfilter sich beim Messen nicht perfekt in der Fourierebene befand. Trotzdem lässt sich bei genauerem Hinschauen das erwartete Muster erkennen.





\begin{figure}[h]
	\begin{subfigure}[c]{0.5\textwidth}
		
		\includegraphics[width=1\textwidth]{Quadratgitter.png}
		\caption{}
		\label{fig:Gitter}
		
	\end{subfigure}
	\begin{subfigure}[c]{0.5\textwidth}
		\includegraphics[width=1\textwidth]{0Grad_Tiefpass_quadrat.png}
		\caption{}
		\label{fig:0Gitter}
	\end{subfigure}
	\caption{In \cref{fig:Gitter} ist das Quadratgitter ohne einen Filter zu sehen; in \cref{fig:0Gitter} mit einem Tiefpassfilter im 0° Winkel}
	\label{Gitter1}
\end{figure}  




\begin{figure}[h]
	\begin{subfigure}[c]{0.5\textwidth}
		
		\includegraphics[width=1\textwidth]{45Grad_Tiefpass_quadrat.png}
		\caption{}
		\label{fig:45Gitter}
		
	\end{subfigure}
	\begin{subfigure}[c]{0.5\textwidth}
		\includegraphics[width=1\textwidth]{90Grad_Tiefpass_quadrat.png}
		\caption{}
		\label{fig:90Gitter}
	\end{subfigure}
	\caption{In \cref{fig:45Gitter} ist das Quadratgitter mit einem Tiefpassfilter im 45° Winkel zu sehen; in \cref{fig:90Gitter} im 90° Winkel.}
	\label{blabla}
\end{figure}  




\subsection{Hochpassfilterung einer Schraube}
Daraufhin wird eine Schraube einer Hochpassfilterung unterzogen. In \cref{Schraube} sind die Bilder mit und ohne Filter zu sehen. In \cref{fig:Schraube_Filter} ist dabei nur noch der Umriss der Schraube zu sehen. Da sowohl der Laser selbst als auch die Schraube selbst nahezu homogen sind, werden dafür fast nur niedrige Frequenzen benutzt. Der Rand der Schraube hat allerdings mehre kleine Kanten, die durch hohe Frequenzen dargestellt werden und deshalb durch den Hochpassfilter durchkommen.

\begin{figure}[h]
	\begin{subfigure}[c]{0.5\textwidth}
		
		\includegraphics[width=1\textwidth]{Schraube_ohneFilter.png}
		\caption{}
		\label{fig:Schraube}
		
	\end{subfigure}
	\begin{subfigure}[c]{0.5\textwidth}
		\includegraphics[width=1\textwidth]{Schraube.png}
		\caption{}
		\label{fig:Schraube_Filter}
	\end{subfigure}
	\caption{In \cref{fig:Schraube} ist die Schraube ohne Filter zu sehen, in \cref{fig:Schraube_Filter} wurde noch ein Hochpassfilter benutzt.}
	\label{Schraube}
\end{figure}  

\subsection{Dunkelfeldmethode}
Zuletzt soll noch die Dunkelfeldmethode getestet, die Luftströme sichtbar machen kann. Dazu wird ein Teelicht 

\begin{figure}[h!]
	\centering
	\includegraphics[scale = 1]{Flamme.png}
	\caption{}
	\label{Flamme}
\end{figure}
